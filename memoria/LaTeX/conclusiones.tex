\chapter{Conclusiones}

Una vez llegados a este punto y finalizado el aprendizaje y desarrollo de aplicaciones es la hora de evaluar el trabajo realizado volviendo a los objetivos marcados en el capitulo dos y valorando en que medida se han conseguido.

\begin{enumerate}
    \item En primer lugar se ha conseguido realizar un manejo bastante amplio de los eventos de Youtube a través de las funcionalidades aportadas por el Youtube API.
    \item Se ha logrado encontrar una tecnología compatible con Youtube API, capaz de retransmitir contenido multimedia hacia sus servidores de forma que el proceso de manejo e inicialización de retransmisiones ha obtenido un alto nivel de automatización.
    \item Una vez conseguidos los dos puntos anteriores se ha podido aunar ambos puntos en una aplicación web, de forma que podemos crear, iniciar o finalizar eventos y retransmisiones a través de dicha aplicación.
    \item Para finalizar el trabajo se ha conseguido desarrollar un adaptador capaz de combinar la tecnología de Youtube con JdeRobot, de forma que podemos retransmitir el contenido captado por una cámara que se encuentra a bordo de un dron.
    
    Como todo proyecto a medida que se avanza con el van surgiendo nuevas ideas y subjetivos de forma que a lo anterior se ha añadido una aplicación web StreamingDron, cuyo objetivo es facilitar el manejo del adaptador ffmpeg a JdeRobot. Otro añadido en el desarrollo del adaptador fue dotarle de autonomía es decir que este adaptador pueda ejecutarse sin necesidad de aplicación web ya que tiene como añadido opcional una interfaz gráfica que muestra las imágenes captadas. También cabe destacar la posibilidad de añadir texto en el vídeo en tiempo real.
    
    Echando la vista atrás y analizando los proyectos antecesores se ha conseguido dar un gran salto en la difusión del contenido ya que en esta ocasión gracias a la inclusión de la tecnología de Youtube disponemos de una capacidad de difusión prácticamente infinita pudiendo llegar este contenido a millones de personas, todo ello en tiempo real.
    
    Como valoración global del proyecto consideramos que los objetivos presentados a principios de él han sido conseguidos de forma satisfactoria como se puede observar en los experimentos llevados acabo a lo largo del desarrollo todos ellos disponibles en la wiki oficial del proyecto \footnote{http://jderobot.org/Apavo-tfg}
    
    Tras el punto y final de este trabajo se abre un gran abanico de posibilidades de cara a futuros proyectos como puede ser la inclusión del audio en el dron ya que a día de hoy este es captado localmente en el servidor.
    Por otro lado y una vez Youtube haya madurado estas características se podría hacer compatible la aplicación web con vídeos 360º o la retransmisión inmediata de contenido sin necesidad de un codificador intermedio.
    Saliendo un poco mas del ámbito del proyecto y centrándonos en el manejo del dron un futuro trabajo podría consistir en dotar a dicho dron de autonomía de forma que no fuera necesario su teleoperación.
\end{enumerate}